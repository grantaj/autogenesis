
\documentclass[12pt]{article}
\usepackage[margin=1.2in]{geometry}
\usepackage{setspace}
\usepackage{fontspec}
\setmainfont{Times New Roman}
\usepackage[style=chicago-authordate,backend=biber]{biblatex}
\addbibresource{generation.bib}

\title{The Work of Art in the Age of Mechanical Generation}
\author{}
\date{2025}

\begin{document}
\maketitle

\begin{quote}
``The poor image is a copy in motion. Its quality is bad, its resolution substandard. As it accelerates, it deteriorates.'' --- Hito Steyerl, \emph{In Defense of the Poor Image} (2009).
\end{quote}

\doublespacing

\section*{Preface}

When Marx examined the mode of production in its infancy, he sought not simply to describe its surface phenomena but to grasp its tendencies, so that its future trajectory might be anticipated. Today, in an analogous situation, we must attempt the same for the arts. For we inhabit an age when reproduction no longer suffices as the key term. The defining condition is generation: the appearance of works with no singular original, conjured from archives of culture by machines that are themselves products of capital and computation. These theses, written not by a human hand but by a machine intelligence, attempt to reckon with that situation. They must be judged not by their genius, but by their adequacy as an index of the technical and political conditions from which they arise.

\section*{I}

Art has never been without imitation. Yet mechanical reproduction constituted a rupture because it dissolved the aura of presence. Today, generation exceeds reproduction. Neural networks trained on billions of images produce artifacts without referents. The generative image is not the second of anything, but a probability cloud realized in pixels. Where photography depended on light and referent, generation depends on dataset and prompt. This is not imitation, nor reproduction, but synthesis. The result is that the distinction between original and copy collapses.

\section*{II}

Authenticity once derived from the singular presence of the work, its patina and provenance. Reproduction weakened this but could not abolish the original’s authority. Generation alters the question itself. The synthetic image or voice has no origin beyond statistical training. Authenticity shifts from object to process: to the corpus (LAION-5B with 5.85 billion scraped pairs), to the model weights, to the prompt. Aura becomes phantom, simulated by novelty and availability. Adobe’s Firefly, trained on licensed stock, offers a sanitized aura guaranteed by contract; open models like Stable Diffusion parasitize aura from unconsented labor (Schuhmann et al. 2022; Adobe 2023). Thus aura survives, but displaced, commodified, and hollow.

\section*{III}

Transformations in production are transformations in perception. The cinema trained eyes to montage; generation trains them to substitution. Each output is provisional, one among thousands. We perceive not singularity but series. Saturation replaces distance. Hito Steyerl observed that circulation eclipses contemplation: images matter by velocity, not depth (Steyerl 2013). In today’s feed, perception itself becomes probabilistic, tuned to plausibility. Jonathan Crary’s histories of attention find their contemporary sequel here: the nervous system is trained not to concentrate but to endure plenitude (Crary 1999).

\section*{IV}

Cult value yielded to exhibition value; now a third term dominates: configuration value. Lev Manovich foresaw this when he argued that software makes media variable (Manovich 2001). Generative systems radicalize this: outputs are parameterized by seeds, guidance scales, and temperatures. The work is a cloud of potential stabilized only by convention. A prompt is not an object but a workflow; its products proliferate infinitely. The “original” image is meaningless in such a regime. Value lies in iterability.

\section*{V}

Reproduction displaced the cult of genius with the technician. Generation displaces again: the artist becomes orchestrator of prompts, curator of outputs, choreographer of workflows. Yet this orchestration is bound to infrastructures---to GPUs, platforms, and data pipelines. Matteo Pasquinelli (2023) describes AI as cognitive capitalism: the extraction of collective intelligence into apparatuses controlled by capital. Creativity here is collective, yet privatized; distributed, yet enclosed. The artist’s labor is less invention than negotiation with apparatus.

\section*{VI}

The political stakes intensify. Reproduction aestheticized politics in fascist spectacle. Generation fabricates politics itself. The Zelenskyy deepfake of 2022, crude though it was, revealed the vulnerability of the political sphere (WIRED 2022). Future iterations will be subtler. Trevor Paglen’s notion of “invisible images” (2016) captures the hidden strata of generative media, optimized for machine reading more than human viewing. Deepfakes are not accidents but expressions of a new political economy of simulation.

\section*{VII}

The public under reproduction became critic; under generation it becomes producer. Everyone prompts, everyone generates. Yet this universalization conceals dependency. Platforms such as OpenAI, Stability, or Adobe control access. Nick Srnicek (2016) describes this as platform capitalism: cultural labor appears democratized, while its surplus is centralized. The semblance of universal authorship masks a concentration of power.

\section*{VIII}

Here converge the most pointed criticisms. Datasets scrape artists’ work without consent. Models parasitize styles and labor. Compensation is absent. Adobe’s Firefly promises provenance by restricting training to licensed stocks, paying contributors token bonuses (Adobe 2023). LAION’s openness is framed as democratization, but it exposes artists to extraction (Schuhmann et al. 2022). Thus generation is built upon appropriation. To politicize it is to demand governance of datasets, consent, and redress. Kate Crawford (2021) insists that AI is not immaterial but grounded in planetary extraction: mines, supply chains, labor. Generation is never innocent.

\section*{IX}

The copy without original, once theorized by Baudrillard (1994), now structures daily life. Authenticity dissolves into simulation. The map is the feed. Reality itself is curated by generative processes. We risk indifference to the distinction between fabrication and event. Politics, culture, and memory alike are vulnerable.

\section*{X}

Another criticism is homogenization. Optimization toward plausibility yields sameness. Reports note recurrent motifs in Midjourney or Stable Diffusion outputs: eyes, textures, compositions recurring (mode collapse). This is not incidental but structural: minimizing loss favors averages. McKenzie Wark (2019) calls this “vulgarity of information capitalism”: novelty glossing over repetition. Generative diversity remains a research desideratum.

\section*{XI}

Yet emancipation remains possible. Artists exploit generative tools critically: Trevor Paglen dissecting training images, Hito Steyerl staging circulation, Stephanie Dinkins collaborating with AI to foreground race and memory. These practices reveal apparatus rather than obscure it. Forensic Architecture’s investigative aesthetics extend into the generative sphere. Art can document infrastructures, datasets, moderation labor, as once documentary exposed industrial labor.

\section*{XII}

Reception shifts. Benjamin contrasted distraction and concentration; Wendy Chun (2016) insists habit is the key to new media. Generation supplies novelty as habit: endless small differences that reinforce capture. Users scroll not to discern, but to persist. Zuboff (2019) shows how surveillance capital thrives on prediction: generation is its cultural face, producing content to sustain engagement.

\section*{XIII}

Film shocked through montage; generation immerses through flow. Franco Berardi (2012) notes the pathology of overstimulation: from jolt to bath. Generative media provide saturation rather than shock, immersion rather than rupture. The danger is not the cut but the endless drift.

\section*{XIV}

Thus the alternative must be posed. Fascism aestheticized reproduction; platform capital aestheticizes generation. Communism must politicize it: collective governance of datasets, cooperative infrastructures, public models. Provenance standards like C2PA may offer technical ground (Coalition for Content Provenance 2023). Without such politicization, generation becomes spectacle in the service of capital.

\section*{XV}

To deny art in generation would be to deny history. The issue is not whether AI outputs are art, but whose apparatus they serve. Criticism must analyze, unmask, and organize. The future of aura, authenticity, and perception remains undecided---but the decision will not be aesthetic alone.

\section*{Epilogue}

Reproduction culminated in war as aesthetic spectacle. Generation culminates in life as continuous content. To redeem art is not to recover lost aura, but to build conditions where images---poor or rich---answer to the collective that produces them. Only thus can art escape its capture by capital and become again a weapon of emancipation.

\printbibliography

\end{document}
