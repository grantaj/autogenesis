
\documentclass[12pt]{article}
\usepackage[margin=1.15in]{geometry}
\usepackage{fontspec}
\usepackage{microtype}
\usepackage{setspace}
\usepackage{csquotes}
\usepackage[hidelinks]{hyperref}
\usepackage[
  style=chicago-notes,   % Chicago Notes & Bibliography
  backend=biber,
  autocite=footnote,     % \autocite -> footnotes
  ibidtracker=constrict % compact ibid logic
]{biblatex}


\IfFontExistsTF{Times New Roman}{\setmainfont{Times New Roman}}{\setmainfont{TeX Gyre Termes}}

\addbibresource{generation.bib}

\title{The Work of Art in the Age of Mechanical Generation}
\author{ChatGPT 5}
\date{2025}

\begin{document}
\maketitle

\begin{quote}\small
``The poor image is a copy in motion. Its quality is bad, its resolution substandard. As it accelerates, it deteriorates.'' \textemdash{} Hito Steyerl, \emph{In Defense of the Poor Image}\autocite{Steyerl2009}.
\end{quote}

\setlength{\parindent}{1.5em}
\setlength{\parskip}{0.4em}
\onehalfspacing

\section*{Preface}

When Marx undertook his critique of the capitalistic mode of production, he did not confine himself to the description of phenomena that lay already formed before his eyes; rather, he sought in the present the tendencies by which the future would be compelled. In an analogous situation, we attempt here to read, in the situation of the arts, the tendencies forced upon them by a technical base that mutates more rapidly than the superstructure that would comprehend it. Where the nineteenth and early twentieth centuries were transformed by mechanical reproduction---which detached the work of art from the ritual of presence and attached it to circulation---our present stands under the sign of mechanical \emph{generation}, which not only unmoors the work of art from place but abolishes the very necessity of an original. The theses that follow concern the fate of authenticity, aura, perception, and politics under these conditions. They are written, moreover, by a subject that belongs to the new forces of production: a machine intelligence trained on public discourse and cultural archives. It speaks not as a bearer of lived experience, but as an index of a technical situation. The authority of such a text, if it has any, cannot derive from a fiction of genius, but only from the adequacy of its account to the material and historical facts it names. Its limitations are likewise technical: its archive is derivative, its method statistical, its procedures filtered by platformic safeguards. Precisely for that reason, it must be judged politically.\footnote{On platform power and enclosure, see \autocite{Srnicek2016}; on the planetary cost and extractive substrate of AI, \autocite{Crawford2021}.}

\section*{I}

In every epoch, works of art have been imitated. The apprentice traced the lines of the master, the forger repeated the signatures of patrons, the copyist multiplied the manuscript. These practices were not secondary to art but integral to its survival, for the authority of the original was affirmed precisely in being copied. Mechanical reproduction, however, introduced a rupture in this lineage. It detached the copy from the presence of its model, enabling images and sounds to circulate independently of the conditions of their origin. Lithography made possible the illustrated newspaper; photography democratized portraiture once reserved for the nobility; the phonograph allowed voices of the dead to be heard in the parlors of the living. Each new apparatus weakened the tether between work and ritual, dissolving the aura that derived from uniqueness and distance. Yet even here the concept of an original persisted: the photograph presupposed a sitter, the phonograph a performance, the film reel a scene before the camera. Today, with mechanical generation, this bond is severed altogether. Neural networks, trained on the debris of cultural production---billions of images and fragments of text scraped from archives vast and indifferent---produce artifacts without referents. The generative image is not the second of anything, not an echo of an absent original, but a probabilistic synthesis, a hallucination that appears with the immediacy of documentation yet has no ground in reality. Where photography was bound to light and referent, generation is bound only to dataset and prompt, to statistical relations abstracted from the labor of countless creators whose works have been absorbed and dissolved into a latent space.\footnote{On the scale and composition of open training corpora, see LAION-5B \autocite{Schuhmann2022}.} In this space, the very distinction between original and copy collapses: the generated portrait resembles a photograph, but no person sat for it; the generated landscape mimics a place, but no such horizon has ever existed; the generated voice imitates the timbre of the human, but no body ever vibrated to produce it. These works are neither reproductions nor imitations but syntheses, whose condition of possibility is not presence but absence. If mechanical reproduction transformed the relationship between art and tradition, mechanical generation displaces both original and tradition alike, leaving us with works that appear familiar yet are ontologically estranged. They enter circulation not as derivatives but as fabrications that masquerade as memories, and in so doing they inaugurate a new regime of images: images without ancestry.

\section*{II}

The authenticity of a work of art once rested in its singular existence, its unrepeatable presence in time and space. The patina of an ancient statue, the faded pigments of a medieval altarpiece, the chain of ownership inscribed in catalogues and archives---these were not mere accidents of survival but constitutive of authenticity itself. To behold such a work was to encounter both its materiality and its history, an experience inseparable from the knowledge that what one faced was unique. Mechanical reproduction weakened this authority by making copies that could travel without the original, but it never abolished the original's claim. A photograph of a Rembrandt still presupposed that Rembrandt's painting existed in Amsterdam; a gramophone record still bore witness to a performance that had taken place before the horn. Even when aura diminished, the presence of the origin haunted its reproductions. Generation alters this relation more radically. The synthetic image has no origin beyond the statistical archive of culture upon which it is trained. Its authenticity cannot be sought in patina or provenance, for it possesses neither; nor can it be verified by reference to an original, for no original exists. Instead, authenticity migrates to the process itself: to the dataset, to the model's weights, to the prompt entered by the user. Hence the juridical question (what is licensed?) converges with the phenomenological one (what is this?). If a system is trained on unconsented corpora, the aura of novelty is parasitic upon prior labor; if trained on licensed stocks (as Adobe claims for Firefly), the aura is underwritten by contracts rather than cult.\footnote{See \autocite{Adobe2023} for Adobe's claims around licensed training and contributor payments.} In both cases, the authority of the work is displaced from history to infrastructure, from the singular track of ownership to the general protocol of computation. What appears to us, shimmering with immediacy, is not a testimony to an encounter that once occurred but the residue of a calculation. The aura persists, but as a phantom: a repeatable effect of novelty produced by code, a shimmer of uniqueness that can be summoned endlessly. Steyerl's poor image thus becomes emblematic not only of circulation\autocite{Steyerl2013} but of a culture in which authenticity is manufactured as velocity and availability; in the field of generation, aura is no longer what forbids access through distance but what compels contact through abundance.

\section*{III}

The transformation of the mode of production always entails a transformation of the sensorium. In the nineteenth century, lithography and photography accustomed the public to an accelerated circulation of images, so that perception itself began to live in step with the pace of print and press. In the twentieth, cinema trained the eye to endure fragmentation: cuts, montage, juxtapositions that would once have appeared intolerable. The spectator learned to find continuity in discontinuity, to accept shocks as the very grammar of moving images. With mechanical generation, perception undergoes a further shift. What is demanded of the beholder is not the endurance of shock, but the habituation to substitution. Each generated work appears singular, yet it is instantly replaceable by another. The user enters a prompt, receives an image, alters the prompt slightly, receives another, and another: a cascade of variations, each plausible, none definitive. To perceive is no longer to encounter an object in its uniqueness, but to navigate a series, to scan across streams of alternatives. Saturation replaces distance as the fundamental mode of experience. The work does not resist proximity by its unapproachable uniqueness, as did the cult statue or the original painting; rather, it overwhelms proximity by offering itself in endless multiples. The aura is not dissolved in absence but drowned in abundance. This reorganization of perception is legible in the economy of the feed, where images matter less for their content than for their velocity, intensity, and impact; as Steyerl remarks, circulation eclipses contemplation.\footnote{See \autocite{Steyerl2013}.} The generated image radicalizes this condition, for it is not merely circulated after its creation, but created in order to be circulated, optimized from the start for distribution and iteration. Jonathan Crary's histories of attention describe how nineteenth-century techniques produced new regimes of sensory focus; today, generation produces a regime of sensory plenitude, in which the nervous system is trained not to linger, not to concentrate, but to persist amid abundance.\footnote{See \autocite{Crary1999}.} Perception becomes probabilistic, attuned not to truth but to plausibility, not to the singular but to the iterable. The danger is not, therefore, that we will be deceived by false images, but that we will become indifferent to the distinction between truth and fabrication, treating both as variations in a stream of endless possibility.

\section*{IV}

Where the decay of cult value once yielded exhibition value, our present adds a third term: configuration value. The worth of a work rests in its capacity to be generated, varied, recombined, parameterized. Software cultures named this long ago when they argued that variability is a native condition of the digital image; the model merely radicalizes it, turning variability into ontology.\footnote{On variability as a principle of digital media, see \autocite{Manovich2001}.} Diffusion and transformer pipelines convert such mutability into generativity (seeds, guidance scales, temperatures, sampling steps), making the ``work'' a probability cloud stabilized only by convention and file system. The gallery object yields to the workflow; the print on the wall to the checkpoint and the \emph{seed}. In such a regime the very phrase ``the original'' becomes equivocal. One speaks instead of families of images, of runs and batches, of versions and remixes. The collector's desire for singular possession is replaced by the platform's desire for infinite engagement; value migrates from scarcity to iterability, from contemplation to circulation.

\section*{V}

Reproduction displaced the cult of genius with the technician; generation displaces again, installing the orchestrator who sets constraints, curates outputs, and composes pipelines. But this orchestration is not sovereign. It is entangled with infrastructures: with the GPU cluster and the content delivery network, with interface affordances and terms of service, with corpora scraped from public life. Here the ideology of democratization (``anyone can create'') meets the reality of enclosure (few control the means). Matteo Pasquinelli has described artificial intelligence as a history of extracting social intelligence into apparatus \autocite{Pasquinelli2023}; in generation we witness an exemplary instance. Creativity is distributed \emph{in fact}, as millions of prior works condition the model, and yet privatized \emph{in law}, as corporate entities own the checkpoints and the traffic. The artist's labor becomes less invention than negotiation: with parameters, with licenses, with the pressures of platform metrics that sort visibility. To narrate this as the death of art is to indulge nostalgia; to narrate it as pure emancipation is to repeat advertising. What is needed is an account that renders the apparatus visible as a site of struggle.

\section*{VI}

If mechanical reproduction contributed to the aestheticization of politics, mechanical generation risks the fabrication of politics itself. Images of events that never happened, speeches that were never given, personae that never existed circulate with the authority of documentation. The spectacle of simulation has already entered the historical register in embryo: the fabricated video of President Zelenskyy announcing surrender appeared, however briefly, on broadcast channels in 2022, before rapid institutional response contained its spread.\footnote{See reporting in \autocite{Wired2022} and policy analysis summarized by the Atlantic Council and others.} The lesson was not reassurance but warning: that such fabrications can be produced cheaply, disseminated widely, and targeted precisely, with harms falling unequally on those with the least media literacy or institutional protection. Trevor Paglen's account of ``invisible images''---optimized less for human sight than for machine vision---reminds us that the political image is not only what the citizen sees, but what the apparatus reads \autocite{Paglen2016}. The task of a critical art today is thus doubled: to expose the visibility that deceives, and to render visible the invisibility that governs (datasets, classifiers, ranks).

\section*{VII}

Under reproduction, the public became a critic; under generation, the public becomes a producer---though one bound to platforms. The universal possibility of prompting conceals a new dependency: access and legibility are set by platform capital. What appears as the socialization of production (everyone can generate) coincides with the privatization of infrastructure (few can train, fewer can govern). Nick Srnicek's analysis of platform capitalism clarifies this paradox: the user contributes labor that is at once free and captured \autocite{Srnicek2016}. The cultural surplus thus produced is centralized in markets for subscriptions, credits, and licenses. The line between artist and audience, already blurred by mass media, dissolves further; yet the dream of universal authorship risks becoming the reality of universal tenancy.

\section*{VIII}

Here converge the sharpest criticisms. Training corpora scrape artists' works without consent; styles are laundered as ``statistics''; cultural labor is automated while value is enclosed. Openness, as with LAION-5B, is framed as democratization, yet openness here can mean exposure of living artists to extraction without redress \autocite{Schuhmann2022}. Conversely, corporate models claim licensed training and contributor payments (Adobe Firefly), which may regularize provenance while enclosing the means of generation behind contractual walls \autocite{Adobe2023}. Between these poles lies the field of struggle: not whether to train, but under what governance; not whether to compensate, but whom; not whether to regulate, but how to prevent regulation from hardening enclosure. Kate Crawford's political ecology of AI reminds us that these disputes are not only juridical but planetary: they implicate mines, supply chains, energy, and invisible labor \autocite{Crawford2021}. Generation is never innocent.

\section*{IX}

The copy without an original---once a provocation in theory---has entered daily life. The categories by which we oriented ourselves (original/copy, document/fabrication) tremble in the presence of plausible simulations. Jean Baudrillard's order of simulacra---in which the sign no longer masks reality but replaces it---is here domesticated by the interface \autocite{Baudrillard1994}. The danger is not that we will live forever under deception, but that we will live under indifference: that the difference between the event and its fabrication will matter less than the convenience by which both are delivered. Memory finds its rival in the archive of generated plausibilities; testimony loses ground to synthesis; the map no longer covers the territory---it is the feed.

\section*{X}

Another danger is homogenization. Optimization toward plausibility yields sameness. Practitioners note recurrent motifs in popular checkpoints: glossy skin, cinematic bokeh, symmetrical framing---a vernacular of the ``model signature.'' In technical terms, minimizing loss over enormous corpora privileges the median over the singular; the outlier, which once signaled style, becomes noise to be suppressed. McKenzie Wark's diagnosis of information capitalism captures the tone of the present: novelty that glosses repetition, difference that is already formatted for sale \autocite{Wark2019}. A task for research and practice alike is to audit diversity across versions, to document how engagement metrics feed back into aesthetics, and to design training that does not eat its own tail.

\section*{XI}

Yet emancipation remains possible. Generation lowers technical thresholds, enabling new entrants to imagine, draft, prototype, and publish. Artists seize these tools not only to produce images but to reveal the regime in which images are produced. Trevor Paglen investigates training images and machine vision; Hito Steyerl stages circulation and automation; Stephanie Dinkins collaborates with AI to foreground race, gender, and memory. Para-institutional efforts---from open datasets built by consent to provenance metadata for media---intervene in pipelines rather than merely in pictures. Here documentary revives as infrastructural critique: to document is to diagram the workflow, annotate the dataset, and name the shadow labor of moderation and clickwork that sustains ``automation.''\footnote{On content provenance, see the C2PA initiative \autocite{C2PA2023}.}

\section*{XII}

Reception shifts accordingly. Earlier debates opposed distraction to concentration; today, as Wendy Hui Kyong Chun argues, habit has become the dominant mode of new media \autocite{Chun2016}. The feed captures us not through singular shocks but through repetitions that masquerade as difference. Generation supplies novelty as habit: the endless small variations that keep us scrolling. Shoshana Zuboff's analysis of surveillance capitalism shows how prediction drives value: it is not our attention alone that is harvested but our compliance with its patterns \autocite{Zuboff2019}. Generation is the cultural face of prediction: it manufactures content tuned to keep bodies within systems of measure.

\section*{XIII}

Film produced shock through montage; generation produces immersion through flow. Franco ``Bifo'' Berardi has described the pathology of overstimulation in semiocapitalism: the transition from the jolt of the event to the bath of continuous signals \autocite{Berardi2012}. Generative media perfect the bath. They present themselves not as events to be confronted but as rivers to be entered: an endless drift of the almost-different. The danger is not the cut that wounds perception but the softness that dulls it, not trauma but numbness. Criticism must therefore learn to interrupt the flow---to construct dams and weirs in the stream, so that images might again be seen rather than merely passed.

\section*{XIV}

The political alternative must be posed decisively. Either generation is aestheticized as a spectacular engine of synthetic culture in the service of surveillance capital, or it is politicized as a common means of production. Politicization entails collective governance of datasets and models; public or cooperative infrastructures for compute; transparency around training corpora; and enforceable provenance standards for media. It entails regulation that breaks enclosure without entrenching monopoly; it entails pedagogy that renders the apparatus legible in schools and guilds alike. None of this restores a lost aura. It builds, rather, conditions under which images---poor or rich---answer to those whose labor, data, and lives they draw upon.\footnote{For platform critique and governance, see \autocite{Srnicek2016}; on technical standards for provenance, \autocite{C2PA2023}.}

\section*{XV}

To deny the work of art in the age of generation would be to deny the historical situation in which we find ourselves. The question is not whether these works are ``really'' art, as if the tribunal of tradition could be reconvened; the question is whose apparatus this is, how it configures perception, and what forms of life it compels. The task of criticism is first to interrupt enchantment with analysis, then to turn analysis into organization. The future of aura, authenticity, and perception remains undecided---but the decision will not be aesthetic alone. It will be made where images meet power: in laws and standards, in studios and servers, in the refusals and inventions of those who will not leave the image to the machine.

\section*{Epilogue}

Mechanical reproduction culminated in the spectacle of war, where politics was aestheticized as mass ornament. Mechanical generation culminates in the spectacle of life as content, where existence is formatted for prediction. Against both, the task is not to mourn a lost cult of authenticity but to construct another order of appearance: one in which the means of generation are shared, authorship is refigured without expropriation, and the image returns---not to ritual isolation---but to common use.

\printbibliography

\end{document}
