
\documentclass[12pt]{article}
\usepackage[margin=1.15in]{geometry}
\usepackage{fontspec}
\usepackage{microtype}
\usepackage{setspace}
\usepackage{csquotes}
\usepackage[hidelinks]{hyperref}
\usepackage[style=chicago-notes,backend=biber,autocite=footnote,ibidtracker=constrict]{biblatex}

\IfFontExistsTF{Times New Roman}{\setmainfont{Times New Roman}}{\setmainfont{TeX Gyre Termes}}

\addbibresource{generation.bib}

\title{The Work of Art in the Age of Mechanical Generation}
\author{}
\date{2025}

\begin{document}
\maketitle

\begin{quote}\small
“The poor image is a copy in motion. Its quality is bad, its resolution substandard. As it accelerates, it deteriorates.”
\footnote{Hito Steyerl, “In Defense of the Poor Image.” See \autocite{Steyerl2009}.}
\end{quote}

\setlength{\parindent}{1.5em}
\setlength{\parskip}{0.35em}
\onehalfspacing

\section*{Preface}

In the second decade of the twenty–first century, an image circulated purporting to show President Zelenskyy announcing surrender. It was fabricated by a generative adversarial network, broadcast briefly, and then exposed.\autocite{Wired2022} The episode was trivial in its failure but monumental in its significance: for the first time, the authority of political speech was shaken not by censorship, but by simulation. Our concern here is not merely with such deceptions, but with the wider transformation of culture that enables them. Just as the printing press produced a world in which images and texts could not be contained within ritual, and just as photography produced a world in which the past could be indefinitely replayed, generative systems are producing a world in which the very categories of origin, originality, and authenticity begin to erode. The task is not to measure what has been lost, but to understand the forms of perception, authorship, and power that are being forced upon us. These theses are written by a subject that belongs to the new forces of production: a machine intelligence trained on collective archives of human making. Their authority, if any, lies not in experience but in the adequacy of their historical and technical account.

\section*{I}

It is no longer sufficient to speak of reproduction. Reproduction presupposes that something precedes it, that there exists a referent whose contours are preserved in its copy. Generation names a different condition: one in which the image has no referent, where the sound has no performance, where the text has no precursor. The dataset replaces the sitter; the prompt replaces the scene. In this way, generative media dissolve the distinction between original and derivative, for the artifact appears without ancestry. To call such works “copies” is to mistake their ontology: they are not copies, but fabrications which only masquerade as documents. They arrive already plausible, already persuasive, and already detached from any origin. If photography still bore the trace of light reflected from a body, generation bears only the trace of statistical correlation abstracted from billions of prior works.\autocite{Pasquinelli2023} The aura, once tied to uniqueness, now flickers strangely: it inheres not in a singular object, but in the uncanny plausibility of a fabrication.

\section*{II}

Authenticity, once rooted in provenance, patina, and the endurance of a work across time, has migrated to infrastructures. When Adobe promotes Firefly as “responsible AI” on the grounds that its models are trained only on licensed stock imagery,\autocite{Adobe2023} it is not the artifact that is authenticated, but the process that produced it. Similarly, open datasets such as LAION frame legitimacy in terms of scale and openness, even as they reproduce the asymmetries of extraction from artists whose works are absorbed without consent.\autocite{Schuhmann2022} What was once the mark of the object is now the brand of the platform. Authenticity has become procedural: to know whether an image is legitimate, one must not examine its surface but its metadata, its training set, its license. This displacement is not neutral. It shifts the ground of authority from the encounter with the work itself to the infrastructures of cloud compute, dataset governance, and platform regulation. The claim of authenticity is no longer that “this object is unique,” but that “this process was approved.” The result is that the very term, once bound to aura, risks becoming indistinguishable from compliance.

\section*{III}

Perception itself is reorganized under generation. Where photography habituated the eye to indexical trace, and cinema trained the body to assemble continuity out of fragments, generative systems require us to navigate endless variation. The user adjusts a prompt, receives an image, alters it slightly, and receives another. Each appears singular, yet each is instantly replaceable. The perceptual regime is not one of shock, as with montage, but of substitution: the normalization of ceaseless alternatives. Saturation, not distance, becomes the dominant mode of experience. In the infinite gallery of outputs, perception learns to slide laterally rather than dwell, to scan across streams of plausibility rather than confront the singular object. This saturation threatens to drown aura not by absence but by excess, not by destroying uniqueness but by offering too many versions of it. The very condition of plausibility becomes aesthetic. As Steyerl remarks, circulation eclipses contemplation; images acquire force by velocity more than by depth,\autocite{Steyerl2013} and generation radicalizes this condition by creating in order to circulate. Crary’s histories of attention remind us that such perceptual habits are historically constructed; generation constructs a habit of lateral scanning rather than sustained encounter.\autocite{Crary1999}

\section*{IV}

If reproduction displaced the cult value of the unique work with the exhibition value of the copy, generation introduces what we might call configuration value. The worth of a work rests less in its stability than in its capacity to be varied, parameterized, reconfigured. The diffusion pipeline is exemplary: images are not produced ex nihilo but as configurations in a latent space, iterable and revisable. What is prized is not the single image but the potential to endlessly produce more. This logic aligns with platform infrastructures themselves, whose economic value derives not from the sale of discrete products but from streams of engagement, continual variation, endless beta. Lev Manovich already identified variability as a native condition of digital media; generation intensifies this condition until variability itself becomes the object.\autocite{Manovich2001} The generated work is not so much a thing as a hinge: a temporary configuration within a system designed for perpetual transformation.

\section*{V}

With this shift comes a new figure of authorship. Reproduction elevated the technician who mastered the apparatus; generation elevates the orchestrator who configures prompts, curates outputs, stitches workflows. Yet this authorship is entangled with infrastructures. The orchestrator is never sovereign: the range of possible outputs is conditioned by the training corpus, by the biases embedded in datasets, by the affordances of user interfaces, by the terms of service of platforms. Here the rhetoric of democratization—“anyone can create”—meets the reality of enclosure: few can train, fewer still can govern. Nick Srnicek has shown how platform capitalism thrives on precisely this contradiction, in which users supply labor under the guise of access.\autocite{Srnicek2016} In generative art, the contradiction takes aesthetic form: the orchestration of difference occurs within a space already determined by infrastructures one cannot touch. Authorship appears distributed, but the distribution masks an asymmetry of power.

\section*{VI}

If reproduction destabilized politics by aestheticizing it, generation unsettles politics by fabricating it. The Zelenskyy deepfake was only a foreshadowing of a broader horizon: the weaponization of synthetic speech, synthetic likeness, synthetic crowds. Here, images and voices no longer testify to events; they summon events that never were. The risk is not merely deception but saturation: an environment in which the distinction between testimony and simulation ceases to orient perception. Jean Baudrillard’s provocation that the sign comes to precede the real is domesticated by the interface,\autocite{Baudrillard1994} as synthetic documentation floods the channels through which publics decide. The appropriate response is neither panic nor complacency, but the construction of provenance regimes and public literacies equal to the task.

\section*{VII}

The economy of art is unsettled accordingly. Generative systems lower the barrier of production to near zero, inaugurating what many describe as a crisis of artistic labour. Works that once required months of craft can be simulated in minutes, and the platforms that host these systems profit not from selling works but from leasing access to possibility. The effect is a double alienation: artists see their labour expropriated into datasets without consent, and audiences encounter works detached from the histories that once oriented judgment. Shoshana Zuboff’s analysis of surveillance capitalism clarifies the logic: prediction is monetized; novelty is instrumentalized; engagement is the commodity.\autocite{Zuboff2019} In this economy, value accrues less to singular works than to the infrastructures that modulate attention.

\section*{VIII}

Yet emancipation remains possible. Artists exploit generative tools critically: Trevor Paglen dissects machine vision and the training image; Hito Steyerl stages circulation and automation; practitioners of adversarial design poison datasets and craft camouflage to reveal the operations of the apparatus.\autocite{Paglen2016,Harvey2022} Museums and curators can respond in kind, shifting from mere display to infrastructural critique: exhibiting datasets, pipelines, and moderation labour alongside outputs; commissioning works that expose the model rather than conceal it; adopting content credentials and provenance standards that let publics see the history of an image’s becoming. The goal is not to restore a lost aura, but to construct conditions in which images—poor or rich—answer to the many who make and live with them.

\section*{Epilogue}

Generation does not abolish art; it transforms its ground. The work of art is no longer secured by singular presence, nor adequately grasped by exhibition alone. It is conditioned by infrastructures—datasets, platforms, protocols—through which authenticity, authorship, and aesthetics are now negotiated. To understand this condition is to see that the central question is not whether these outputs are “really” art, but to whom the apparatus belongs, how it configures perception, and what forms of life it compels. Against capture, the task is to politicize generation: to seize the means of seeing.

\printbibliography

\end{document}
